\documentclass{article}
\usepackage{graphicx}
\usepackage[margin=1in]{geometry}
\usepackage{times}
\usepackage{hyperref}

\title{Neuromorphic BCI: Baseline Study with Synthetic Spiking Data}
\author{AI Scientist}
\date{\today}

\begin{document}
\maketitle

\begin{abstract}
We present a minimal baseline for neuromorphic Brain-Computer Interfaces using synthetic spiking data and lightweight decoding. This template is intended for rapid iteration and safe experimentation without clinical data.
\end{abstract}

\section{Introduction}
Neuromorphic BCI focuses on event-based neural signals and efficient decoders under tight latency and power constraints. We simulate Poisson-like spikes with class-dependent spatial and temporal structure to provide a toy yet informative benchmark for method development.

\section{Methods}
We generate synthetic spike trains over channels and time bins. Rate parameters vary by class across channels and time to create separable structure. A linear decoder is trained on aggregated spike features.

\section{Results}
The baseline reports accuracy and training losses. Plots such as spike rasters and learning curves are included.

\section{Discussion and Limitations}
This template uses only synthetic data and is not intended for clinical use. It provides a convenient sandbox for comparing coding schemes and decoders, and for exploring robustness to noise or dropout.

\section{Conclusion}
The template offers a simple starting point for neuromorphic BCI research automation in AI Scientist.

\bibliographystyle{plain}
\bibliography{references}
\end{document}
