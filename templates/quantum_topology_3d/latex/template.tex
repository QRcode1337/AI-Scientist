\documentclass{article}
\usepackage[margin=1in]{geometry}
\usepackage{graphicx}
\usepackage{hyperref}
\title{Exploring Toy 3D Topological Error Correction Under CPU Constraints}
\author{AI Scientist}
\date{\today}
\begin{document}
\maketitle

\begin{abstract}
We explore toy 3D topological quantum error-correction behavior using small, CPU-friendly simulations. Our baseline studies threshold-like behavior via logical error rates (LER) computed from synthetic defect fields and simple decoders.
\end{abstract}

\section{Introduction}
We consider simplified 3D lattices with i.i.d.\ noise that generate defect fields (toy proxies for syndromes). The focus is on qualitative trends and decoder heuristics suitable for fast iteration.

\section{Methods}
We generate 3D binary grids with error probability $p$, identify connected clusters, and estimate logical error rates via spanning-cluster probability across axes. This provides a proxy for threshold-like behavior at small sizes.

\section{Results}
We report LER vs.\ $p$ curves for small lattice sizes and visualize sample configurations via slice montages.

\section{Discussion}
We discuss trade-offs between runtime and estimation quality, and outline future directions (bias-aware heuristics, finite-size scaling, slice-wise decoders).

\section{Limitations}
This is a toy model; results are not hardware claims and do not represent any specific code implementation.

\end{document}
